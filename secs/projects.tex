% Awesome Source CV LaTeX Template
%
% This template is originally maintained by Christophe ROGER from:
% https://github.com/darwiin/awesome-neue-latex-cv
%
% This modification has been download from:
% https://github.com/innerTide/awesome-neue-latex-cv-extended
% Author:
% Yuefeng Wu
%
% Template license:
% CC BY-SA 4.0 (https://creativecommons.org/licenses/by-sa/4.0/)
% Section: Projects
% Usage:
% \project
%  {<End date>}      {<Project Title>}
%  {<Start date}     {
%                      <Contribution to the project>
%                    }
%                    {<Technology list>}
%
%


\sectionTitle{\textsc{Project Experience}}{\faCode}
\begin{projects}

  \project
    {June 2016} {All Digital Phase-Locked Loop (ADPLL)}
    {May. 2016} {
                      \begin{itemize}
                        \item Course project of Digital RF
                        \item Learned knowledge about digital RF and frequency synthesis
                        \item Built a time-domain model of the ADPLL based on its phase operation
                        \item System-level design with the model for behavior-level noise simulation in using MATLAB
                      \end{itemize}
                    }
                    {Digital RF, ADPLL, Frequency Synthesis, Modeling, MATLAB}
  \emptySeparator


  \project
    {May. 2016}   {Transistor Fabrication}
    {Apr. 2016} {
                      \begin{itemize}
                        \item Course project of IC-technology lab.
                        \item Learned basic CMOS fabrication steps and their physical mechanisms with a 1-μm Bi-CMOS process in Else Kooi Lab at Delft University of Technology
                        \item Simulation and hands-on operation of the fabrication of MOS transistors in a clean room
                        \item Lab measurement of the fabricated transistors using a microscope and a probe station
                      \end{itemize}
                    }
                    {CMOS process, Simulation, Fabrication, Clean Room, Measurement}
  \emptySeparator

  \project
    {Mar. 2016}   {Audio Amplifier Design}
    {Feb. 2016} {
                      \begin{itemize}
                        \item Course project of Analog CMOS Design
                        \item Designed a class-AB amplifier architecture to handle a low-ohmic load with a rail-to-rail swing
                        \item Implemented the amplifier using LTspice in 0.18-μm CMOS, and it achieved a high SNR, a high SFDR, and a low IM3
                        \item Optimized the gain and phase margins of the amplifier to achieve a stable operation within the given bandwidth
                      \end{itemize}
                    }
                    {Analog Design, Amplifier Design, Low Noise, Stability Margins, LTspice}
  \emptySeparator

  \project
    {Jan. 2016}   {Time-to-Digital Converter (TDC)}
    {Dec. 2015} {
                      \begin{itemize}
                        \item Course Project of Digital IC Design
                        \item Designed a 10-bit TDC with a 4-bit delay line and a counter to save area and power
                        \item Implemented the TDC using Cadence in UMC 90-nm CMOS, and it achieved a worst-case 27-ps resolution with both DNL and INL smaller than 1 at every process corner
                        \item Optimized the area of the TDC in the layout
                      \end{itemize}
                    }
                    {Mixed-Signal Design, TDC, Corner Simulation, Layout, Cadence}
  \emptySeparator

  \project
    {Dec. 2015}   {Low-Noise Amplifier (LNA)}
    {Nov. 2015} {
                      \begin{itemize}
                        \item Course project of Microwave Circuit Design
                        \item Learned impedance matching, stability, and noise figure in microwave/RF designs
                        \item Designed a CMOS LNA with an inductive degeneration for simultaneous input noise and impedance matching using ADS
                        \item Designed the corresponding output matching network of the LNA to achieve stability
                      \end{itemize}
                    }
                    {Microwave/RF Design, LNA, Matching Network, Stability, ADS}
    \emptySeparator

    % \project %add name of the algorithm used!!!!!!!!!!!!
    %   {June 2014}   {Bachelor Thesis: Design of An Object Tracking Software}
    %   {Dec. 2013} {
    %                     \begin{itemize}
    %                       \item Conducted research on various object tracking algorithms
    %                       \item Designed a Windows software that can track multiple objects in a video file with OpenCV libraries using Visio Studio
    %                       \item Optimized the software to handle real-time camera input for demonstration purposes
    %                     \end{itemize}
    %                   }
    %                   {C++, Object Tracking, Windows, OpenCV, Visio Studio}
    %   \emptySeparator

  \project
    {May. 2014}   {Micro-Electro-Mechanical System (MEMS) Magnetic Sensor}
    {Mar. 2013} {
                      \begin{itemize}
                        \item Student research project at Ministry of Education Key Lab. of MEMS, funded by National Science Foundation of China (No. 61201032)
                        \item Designed a two-dimensional MEMS magnetic sensor in collaboration with other students using ANSYS
                        \item Simulation and optimization of the sensor under various electrical and magnetic (EM) conditions and at process corners
                        \item Resulted in two patents
                      \end{itemize}
                    }
                    {MEMS, Sensor, Leadership, Modeling, Simulation, ANSYS}
  %   \emptySeparator
  %
  % \project
  %   {Apr. 2013}   {Smart Car based on Camera Input}
  %   {Mar. 2013} {
  %                     \begin{itemize}
  %                       \item A prototype for the 7\textsuperscript{th} Freescale Smart Car competition at Southeast University
  %                       \item Built a smart car model with two PWM motors for driving and steering, a greyscale camera for capturing road information, and a microcontroller for overall control
  %                       \item Designed the PCB for the microcontroller in collaboration with my teammates
  %                       \item Implemented the PID motor control part of the car control software in μC/OS-\Rmnum{2}
  %                     \end{itemize}
  %                   }
  %                   {C, μC/OS-\Rmnum{2}, PCB Design, Altium Designer, Team Collaboration}
  %   \emptySeparator
  %
  %   \project
  %     {Dec. 2012}   {Digital Altimeter}
  %     {June 2012} {
  %                       \begin{itemize}
  %                         \item Funded by Student Research Training Program of Southeast University
  %                         \item Designed the altimeter hardware with a microcontroller PCB, a barometric sensor, a temperature sensor, and a LCD display
  %                         \item Implemented the altimeter software using barometric information with temperature error correction in μC/OS-\Rmnum{2}
  %                       \end{itemize}
  %                     }
  %                     {C, μC/OS-\Rmnum{2}, Sensor, PCB Design, Altium Designer}

\end{projects}
